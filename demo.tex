% Demo for commonly used LaTeX macros.

% preamble
\documentclass[10pt,letterpaper]{article}

%\usepackage[T1]{fontenc}
%\usepackage[latin1]{inputenc}
%\usepackage{mathptmx}
%\usepackage{mathpazo}
%\usepackage{newcent}
%\usepackage{bookman}
%\usepackage{helvet}
\usepackage{amsmath}
\usepackage{amssymb}
\usepackage{latexsym}
%\usepackage{eufrak}
%\usepackage{eucal}
%\usepackage{geometry}
%\usepackage{layout}
%\usepackage{array}
\usepackage{amsthm}
\usepackage{graphicx}
\usepackage{xcolor}
%\usepackage[hidelinks]{hyperref}
%\usepackage[final]{listings}

\usepackage{shmacro}
\usepackage{shlang}
\usepackage{shmath}

\vfuzz2pt % allow overflow by 2 points
\hfuzz2pt % allow overflow by 2 points

\title{<document title>}
\author{%
  Spencer Hubbard\\
  College of Arts and Science\\
  University of Washington
}
\date{\today}

\begin{document}
% front matter
\maketitle

% main matter
\section{General Macros}
\label{sec:macro}

\NOTE{This is a note.}

\TODO{This is a todo note.}

\section{Language Macros}
\label{sec:lang}

\begin{figure}[htb]
  \begin{syn}
    x \in Var & & & \textit{any variable} \\
    t \in Term
    & ::= x & \textit{reference} \\
    & | & \LCabs{x}{T}{t} & \textit{abstraction} \\
    & | & \LCapp{t}{t} & \textit{application} \\
  \end{syn}
  \caption{Syntax for lambda calculus}
  \label{fig:syn}
\end{figure}

\begin{figure}[htb]
  \begin{align*}
    &\inferrule[E-AppAbs]
    { }
    { \LCapp{(\dsep\LCabs{x}{T}{t}\dsep)}{v} \eval\ \LCsub{x}{v}{t} } &
    &\inferrule[E-App1]
    { t_1 \eval\ t_1^\prime }
    { \LCapp{t_1}{t_2} \eval\ \LCapp{t_1^\prime}{t_2} } &
    &\inferrule[E-App2]
    { t_2 \eval\ t_2^\prime }
    { \LCapp{t_1}{t_2} \eval\ \LCapp{t_1}{t_2^\prime} }
  \end{align*}
  \caption{Evaluation rules for lambda calculus}
  \label{fig:eval}
\end{figure}

\section{Math Macros}
\label{sec:math}

The set of even integers is defined as
$$
  2\mathbb{Z} = \setb{2k}{k \in \mathbb{Z}} .
$$

The Klein 4-group (or \emph{Viergruppe}) has the following presentation:
$$
  V = \genb{a, b, c}{a^2 = b^2 = c^2 = 1} .
$$

% back matter

\end{document}
